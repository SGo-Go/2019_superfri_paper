\section{Future work}
The proposed benchmark gives a good evaluation tool for a relatively automatic way of proceeding with tests and receiving results which will directly allow using the best HPC architecture.  It means that finally the end user or the resource owner may have different criteria to fulfil their requirements. The resource owner will focus on parameters which are efficient globally (all applications running on the machine), cost efficiency (TCO shown as CAPEX and OPEX, i.e. the investment costs vs. maintenance costs of the HPC). The end user will, however, concentrate on the fastest way to receive results and the most efficient way of parallelisation.

From that point of view the benchmark could be extended by testing automatically
the scalability of the e-Infrastructure and the energy consumption of the running benchmark. A final step of the benchmark could interpret the results for both groups of users and propose the best HPC system in terms of size and architecture (CPU, memory size, aggregated speed to external memory, if necessary).