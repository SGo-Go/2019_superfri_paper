\subsection{Measuring and reporting performance technique}
\label{sec:measuring}

In the process of preparing and launching the benchmark,
we faced with a number of technical difficulties,
which significantly influenced the approach
we have chosen to measure and report performance.
This subsection highlights our approach to tackling those difficulties.

Since the analyzed testbeds represent brand new architectures,
we encountered a limited number of tools for measuring metrics of interest
which were strait available for all testbeds.
In particular,
many popular performance measuring tools -- VampirTrace, Scalasca, etc --
were not ported to the ARM Hi1616 by the time of performing benchmark.
Moreover, tested applications are written in different languages -- C/C++, Fortran, and Python --
which reduces the range of performance measuring tools suitable for all applications at once.
On the other hand, our study is focused on overall performance of the code
and does not require instrumentation to measure and analyze performance.
Thus, we decided to omit specialized benchmarking libraries (e.g., \textsf{LibSciBench} \cite{2015:Hoefler})
and performance analysis tools (e.g., VampirTrace)
in favor of standard Linux toolset available out-of-the-box.
The metrics of interest were measured by means of \texttt{/usr/bin/time} Linux utility.
This utility allows to measure the following metrics for the application as a whole and for each separate MPI process:
total elapsed real time, number of filesystem inputs and outputs,
maximum resident set size of the process,
average total (data+stack+text) memory use of the process,
etc.
We used measurements of the above-mentioned metrics and the data from Table\,\ref{tab:clusterconfig}
to compute dependent metrics (like TDP and TCO) and to build all charts and plots in this paper.
In order to keep conditions of the experiments even, % (cold caches),
the system caches were flushed by calling
\texttt{sync; echo 3 > /proc/sys/vm/drop\_caches} before each experiment.
All C/C++ and Fortran codes were compiled with gcc version 5.4.0 using -O3.
Python 3.5.2 was used as a Python interpreter for the ABM4Py application.

%% http://coegss.eu/wp-content/uploads/2018/02/D5.7.pdf
%% http://coegss.eu/wp-content/uploads/2018/11/D5.8.pdf

In addition, we experienced issues with different orders of magnitude in elapsed times of the applications.
Even though we adjusted the input files to keep the elapsed time scales closer for all application,
we still had significant differences in total elapsed times between social simulation and CFD software:
each individual benchmark for social simulation codes consumed less than 3 hours on any testbed,
while benchmarks for some CFD application required more than 30 hours.
The latter fact prevented us from performing many repetitions of time-consuming CFD application runs.
More specifically, in order to reduce the number of repetitions, for each test configuration
we started with 2 runs and repeated the experiment % proceed with new repetitions of the experiment
until the ratio of the sample standard error to the sample mean of the elapsed time was less than 5\%.
Since the number of measurements generated with this approach
is sometimes insufficient to construct meaningful confidence interval as suggested in the performance reporting guidelines\,\cite{2015:Hoefler},
Figures\,\ref{fig:socapps_plots} and~\ref{fig:cfdapps_plots} report sample means of the measurements without confidence interval.

Last but not least,
our testbeds were limited by 1-2 nodes,
which is not enough to conduct full scalability experiments with some of the selected applications.
In order to ovecome this obstacle,
as a reference testbed we used the Eagle cluster equipped with 50 nodes containing 2 14-core Intel Haswell E5-2697 v3 CPUs each,
where we performed tests up to reaching the scalability bound.
It allowed us to get an impression on the scalability of the applications reflected in Tables\,\ref{tab:bottlenecks_hardware} and~\ref{tab:bottlenecks_cfd_hardware}.

%% 5. availability of testbeds: some testbeds were available for a short time (couple of months);
%% Last but not least, we did not manage to port HWRF to ARM Hi1616 and IBM Power8+ testbeds.

