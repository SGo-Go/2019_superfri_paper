\begin{abstract}
\label{sec:abstract}
The work undertaken in this paper was done in the Centre of Excellence for Global Systems Science (CoeGSS), an interdisciplinary project, funded by the European Commission. The project provides decision-support in the face of global challenges (e.g. the network structure of the world economy, energy, water and food supply systems, the global financial system or the global city system, and the scientific community). It brings together HPC and global systems science. 
This paper presents a proposition of GSS benchmark with the aim to find the most suitable HPC architecture and the best HPC system which allows to run GSS applications effectively. 
The outcome of the analysis is defining a benchmark which represents the average GSS environment in a good way. Three exemplary challenges were defined as pilot applications: Health Habits, Green Growth and Global Urbanisation extended with additional applications from GSS ecosystem. They have been tested on a small set of new HPC platforms based on Intel Xeon Gold 6140, AMD Epyc, ARM Hi1616 and IBM Power8+ processors. 
Results of the tests provide the architecture comparison for different applications based on execution times, TDPs\footnote{Thermal Design Power} and TCO\footnote{Total Cost of Ownership} as the basic metrics used for providing a ranking of HPC architectures. Finally, the analysis or the results is thought to be valuable information for the GSS community for future purposes to determine their specific demands as well as - in general - to help develop a mature final benchmark set reflecting the GSS environment requirements and speciality.
\keywords{Global Systems Science, HPC benchmarks, parallel applications, 
e-Infrastructure evaluation}
\end{abstract}